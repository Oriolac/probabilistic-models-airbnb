\begin{document}
En el present document, s'explicaran els passos seguits per al desenvolupament de les diferents parts de la pràctica de models probabilístics, utilitzant l'eina \texttt{weka}.\\\\
Els objectius principals que s'han volgut assolir, han sigut bàsicament tres: parsejar les dades donades de tal manera que weka pogués llegir-les, crear diferents models de tres tipus diferents: 
\begin{enumerate}
	\item Xarxa bayesiana obtinguda amb K2 amb el paràmetre $u=3$, sent aquest el nombre màxim de pares per variable.
	\item Naive Bayes sent la variable independent \texttt{overall\_satisfaction} i amb $u=0$.
	\item Naive Bayes amb la mateixa variable independent que la classe 2, però sent $u=3$.
\end{enumerate}
D'aquests, s'havia de veure quin dels 10 models de cada classe era millor\footnote{En cas que es puguin tenir 10 mdoels, ja que en el cas de la classe 2 això no era possible degut a que, ab aquesta configuració, només existeix un únic model.}. I, finalment, l'últim objectiu consistia en l'explicació de com s'han escollit els conjunts discrets de valors per a transformar els atributs d'entrada en valors reals.\\
\\
Per a executar el programa es necessari tenir \texttt{python3} al sistema. Si es té,
s'ha de realitzar les següents comandes al terminal en bash:
\begin{lstlisting}[language=bash]
python3 -m venv venv
source venv/bin/activate
python3 -m pip install -r requirements.txt
\end{lstlisting}
La primera comanda crea un \textit{entorn virtual} perquè les llibreries s'instal·lin en l'àmbit de projecte. D'aquesta
forma, per eliminar-les només caldrà eliminar la carpeta. La segona ordre en el terminal 
indica que s'utilitzi l'entorn virtual i, la tercera, instal·la els requeriments que s'han 
utilitzat (són els requeriments de la llibreria \texttt{panda}). Finalment, per a executar el fitxer
de parseig, s'ha d'executar mitjançant la comanda següent\footnote{Aquest és la comanda general, en el nostre cas, la comanda correcta que s'ha d'executar es troba escrita al final de l'apartat 2 d'aquest document.}:
\begin{verbatim}
	python3 -m src <input-file> <train-file> <test-file>
\end{verbatim}
El parser accepta més paràmetres opcionals, com per exemple, per a canviar la llavor utilitzada,
que són discutits en el següent apartat. De tota manera, es pot executar la següent comanda per
a saber que realitza cada un dels paràmetres opcionals:
\begin{verbatim}
	python3 -m src --help
\end{verbatim}
Finalment, s'ha executat el programa amb certs paràmetres opcionals, que són discutits en 
l'apartat 2.3.
\end{document}