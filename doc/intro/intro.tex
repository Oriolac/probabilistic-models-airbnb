\begin{document}
En el present document, s'explicaran els passos seguits per al desenvolupament de les diferents parts de la pràctica de models probabilístics, utilitzant l'eina \texttt{weka}.\\\\
Els objectius principals que s'han volgut assolir, han sigut bàsicament tres: parsejar les dades donades de tal manera que weka pogués llegir-les, crear diferents models de tres tipus diferents: 
\begin{enumerate}
	\item Xarxa bayesiana obtinguda amb K2 amb el paràmetre $u=3$, sent aquest el nombre màxim de pares per variable.
	\item Naive Bayes sent la variable independent \texttt{overall\_satisfaction} i amb $u=0$.
	\item Naive Bayes amb la mateixa variable independent que la classe 2, però sent $u=3$.
\end{enumerate}
D'aquests, s'havien de veure el millor de 10 exemples\footnote{En cas que es puguin tenir més de 10. En el cas de la classe 2, no es poden realitzar 10 exemples, ja que és cas únic.}, i, finalment, l'explicació de com s'han escollit els conjunts discrets de valors per a transformar els atributs d'entrada en valors reals. 
\end{document}