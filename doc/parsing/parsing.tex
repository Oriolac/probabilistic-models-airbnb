\begin{document}
% Es divideix en dos fitxers a src, __main__.py que parseja les dades
% weka.py -> te funcions per a categoritzar les dades i per a crear els fitxers
% argparser es el notre amic
% explicar tots els tipus de paràmetres que podem passar
% explicar el cut i el qcut
% df[column_name] = pd.qcut(df[columnname], q=4, labels=False)
% utilitzem sample i drop per a crear els dos datasets 
%% Les dues funcions pandas.cut i pandas.qcut
El parseig de les dades es realitza en dos fitxers continguts en \texttt{src}:
\begin{itemize}
	\item \verb|__main__.py| conte la funció \texttt{main} i funcions de parseig dels
	arguments passats per la línia de comandes.
	\item \texttt{weka.py} conte funcions auxiliars per a categoritzar les dades i per a crear els
	texts que amb format arff per a weka.
\end{itemize}
\subsection{Linia de comandes}
Per a parsejar les dades passades per la línia d'ordres, donat que el projecte semblava
requerir bastants arguments, s'ha utilitzat el mòdul estàndard d'\texttt{argparser}. Aquest
ens permetia posar arguments opcionals per a ser utilitzats en el parseig. Així doncs, els
diferents arguments que es poden passar són:
\begin{itemize}
	\item \texttt{input-file, train-file, test-file} són els arguments requerits per l'enunciat de la pràctica.
		Input file és l'únic fitxer que ha d'estar creat amb anterioritat i ha de contindre un csv amb els
		valors que es volen parsejar. 
	\item \texttt{pri, rev, lat, lon}: són diferents paràmetres que permeten canviar el nombre de particions en els arguments \textit{price, reviews, latitude i longitude} respectivament. Per a
	canviar el valor, s'ha d'especificar un enter després del parametre, per exemple:
	\verb|python3 -m src ... -pri 3|.\\
	Aquests paràmetres han servit per a provar diferents valors en com partir les dades pel weka.
	En la següent secció s'explicarà de quina manera s'ha realitzat i el perquè d'aquesta.
	\item \texttt{cpri, crev, clat, clon}: Dins de les maneres que es poden dividir en classificacions 
	diferents dades contínues, hi ha dues formes que destaquen: fer la divisió per centils o realitzar-la
	amb rangs de valors de la mateixa mida. La diferència de realitzar-la per quantils és que tots
	els rangs tindran el mateix nombre de valors\footnote{Realment, si la divisió per quartils no
	és exacta, un conjunt podria tenir menys valors, és a dir, si es divideix entre quatre conjunts un
	conjunt de 13 valors, llavors un d'aquests inevitablement tindrà un valor de més.}. Però, dividir-la
	així pot deixar en un mateix rang valors forans i valors normals, és a dir, si es dividís en dos
	grups la llista [2, 3, 4, 2000], 4 i 2000 aniran al mateix grup quan aparentment tenen poc sentit
	que ho siguin. Amb aquest paràmetre permetem en executar l'script, canviar la funció de tall
	d'un tall per rangs amb la mateixa mida als creats a partir de quartils.
	\item \texttt{name}: permet canviar el nom de la relació amb què es guarda. S'ha d'especificar
	en el paràmetre: \verb|python3 -m src ... -n notmydataset|.
	\item \texttt{seed}: permet canviar la llavor en la qual s'agafen els valors per a fer els dos
	datasets (train i test). Si no s'especifica, s'agafa el valor per defecte dels últims 5 dígits d'un
	dels autors de la pràctica, com s'especificava a l'enunciat.
\end{itemize}
 Dins d'aquest codi s'utilitzen dos funcions:
 \begin{itemize}
 	\item \texttt{ArgumentParser}: constructor de la classe. Ens permet crear un parser on l'hi
 	podem atribuir una descripció que s'utilitzarà amb el paràmetre \textit{help} opcional.
 	
 	\item \verb|add_argument|: ens permet afegir un argument. Si comença amb el caràcter
 	'-', serà opcional, i es podrà especificar un nom verbós, com per exemple \texttt{-n} i
 	\texttt{--name} realitzen la mateixa funció. Aquest mètode accepta diferents tipus de 
 	paràmetres per a canviar el seu comportament. A continuació s'expliquen els que s'han utilitzat
 	en aquesta:
 		\begin{itemize}
 			\item \texttt{metavar}: especifica una variable que s'utilitzarà per a mostrar el missatge
 			d'ajuda per a paràmetres no opcionals.
 			
 			\item \texttt{type}: ens permet especificar com s'ha de parsejar els valors per a poder
 			ser utilitzats en el codi. En el nostre cas, només s'utilitza \texttt{str} i \texttt{int}.
 			
 			\item \texttt{help}: proporciona un text d'ajuda quan es realitza la comanda \texttt{--help}.
 			
 			\item \texttt{dest}: variable amb la que serà guardada en el parseig.
 			
 			\item \texttt{action}: especifica una acció per a realitzar. Només s'ha utilitzat per a 
 			guardar una constant \textit{'store\_const'}.
 		
 			\item \texttt{const}: especificar la constant a guarda quan s'utilitza \verb|'store_const'|.
 			Amb conjunció amb el valor \texttt{default} i amb \texttt{action}, ens permet canviar 
 			la funció a utilitzar pel tall.
 		\end{itemize}
 \end{itemize}
\subsection{Weka i pandas}
	Per a realitzar la funció de parseig s'ha utilitzat pandas. Per a fer-ho, s'ha utilitzat la funció
	\verb|read_csv|.  A partir d'aquí, la manipulació de dades s'ha realitzat amb conjunció amb 
	numpy. Així doncs, per a realitzar la discretització de les dades contínues, s'ha utilitzat les
	funcions \texttt{qcut} i \texttt{cut}, la primera fent-ho amb quartils. Aquestes dues funcions
	accepten els mateixos paràmetres, perquè el canvi es pot realitzar fluidament. A més a més
	per a fer el canvi d'una columna continua, es pot realitzar amb:
	\verb|df[column_name] = cut_func(df[column], number_divisions, labels=False)|. 
	\textit{Labels} permet canviar el nom en el qual es categoritzen, passant dels rangs a noms
	donats com a llista. Per facilitar el parseig, s'ha passat el valor \texttt{False}, que canvia simplement numera les instàncies. \\
	\\
	A més a més, s'ha realitzat la funció de \verb|parse_weka|, que retorna una \texttt{str} 
	representant la informació d'arff. S'ha retornat el valor en comptes de guardar-lo directament
	en el fitxer per separar la lògica d'on i que es guarda. La funció en si és una agrupació de 
	\texttt{join} de les files i columnes per les diferents línies. La divisió dels datasets es realitza
	en la mateixa funció per assegurar que els dos sempre tindran el mateix nom a la relació i els
	atributs.
\end{document}