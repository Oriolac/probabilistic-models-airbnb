% !TeX spellcheck = en_US
\begin{document}
%TODO Una explicación de como ejecutar vuestro script y el valor que poner en la función random.seed() (los cinco últimos dígitos de vuestro DNI) para obtener los dos ficheros ARFF que habéis usado para aprender y evaluar vuestros modelos.

% Gràfics de cada categoria i perquè s'ha utilitzat
% price -> cut, num_quart: 15
% reviews -> cut, num_quart: 10 
% latitude -> cut,numQ : 20
% longitude -> cut,numQ : 20
Per tal de passar els atributs que tenen valors reals a valors discrets, hem utilitzat la llibreria pandas, que ens permet llegir el csv d'una manera molt còmode i partir-lo de diferents maneres. Entre aquestes maneres, hem vist que dos mètodes eren bastant bons:
\begin{itemize}
	\item \textbf{cut}. Divideix el dataset en diferents intervals utilitzant el rang de valors.
	\item \textbf{qcut}. Divideix el dataset mitjançant quantiles, és a dir, tots els intervals tindran el mateix nombre de dades.
\end{itemize}
Per tal d'arribar a la conclusió de quin mètode era el millor per tal de categoritzar els atributs, s'ha utilitzat la llibreria \texttt{matplotlib} per tal de visualitzar les dades de manera gràfica. S'ha realitzat un gràfic \footnote{Es poden trobar a la secció \ref{app:grafics} de l'apèndix.}per cada atribut a categoritzar sobre l'\textit{overall\_satisfaction} i després categoritzant l'atribut. Aquest procediment s'ha realitzar bastants cops fins trobar una partició que fos representativa. 

\begin{table}[H]
	\centering
\begin{tabular}{ccc}
	Atribut &Mètode & Num. Divisions\\\hline
	price & cut & 15 \\
	reviews & cut & 10 \\
	latitude & cut & 20 \\
	longitude & cut & 20 \\
\end{tabular}
\caption{Mètodes de partició utilitzats segons cada atribut}
\end{table}

\end{document}